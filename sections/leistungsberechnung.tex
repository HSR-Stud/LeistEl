\section{Leistungsberechnung}
\begin{tabular}{|l|l|}
  Momentane Leistung
  	& $p(t) = u_{R}(t) \cdot i_{R}(t)$\\
  Wirkleistung
  	& $P = \frac{1}{2\pi}\int_{0}^{\pi}\frac{u_{R}^2(\alpha)}{R} \cdot d\alpha = \frac{U_{R RMS}^2}{R}$\\
  Wirkleistung (Trafoseitig)
  	& $P = U \cdot I_{1} \cdot cos(\varphi_{1})$ \newline
  		dabei ist $I_{1}$ die erste Harmonische Komponente des Stromes \newline
  		und $\varphi_{1}$ die Phasenverschiebung \\
  Grundschwingungsblindleistung
  	& $Q_{1} = U \cdot I_{1} \cdot sin(\varphi_{1})$\\
  Verzerrungsleistung
  	& $Q_{V} =  U \cdot \sqrt{\sum_{k = 2}^{\infty}I_{k}^2}$\\
  gesamte Blindleistung
  	& $Q = \sqrt{Q_{1}^2 + Q_{V}^2}$\\
  Grundschwingungsscheinleistung
  	& $S_{1} = U \cdot I_{1}$\\
  gesamte Scheinleistung
  	& $S = U \cdot I_{RMS} = \sqrt{P^2 + Q^2} = \sqrt{P^2 + Q_{1}^2 + Q_{V}^2}$\\
  Leistungsfaktor
  	& $\lambda = \frac{P}{S}$\\
  Welligkeit
  	& $w = \frac{F_{RMS}}{F_{AV}}= \frac{\sqrt{\sum_{k = 1}^{\infty}F_{k}^2}}{F_{AV}}$\\
\end{tabular}