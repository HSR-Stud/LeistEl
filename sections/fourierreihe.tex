\section{Berechnung der Fourierreihe}
\subsection{Orthogonalitätsbeziehungen}

            $\int\limits_0^T \cos(n\omega t)\cdot \cos(m\omega t)dt=
            \begin{cases}
            T,\ n=m=0\\
            \frac{T}{2},\ n=m>0\\ 
            0,\ n\neq m\\
            \end{cases}$\\
            
            
           $\int\limits_0^T \sin(n\omega t)\cdot \sin(m\omega t)dt=
           \begin{cases}
           \frac{T}{2},\ n=m\\
           0,\ n\neq m\\
           \end{cases}$\\
           $\int\limits_0^T \cos(n\omega t)\cdot \sin(m\omega t)dt=0$
           
\subsection{Allgemeine Form}
Eine periodische Funktion lässt sich durch eine Reihe von Sinus- und Kosinusfunktionen darstellen.
$$f(t) = \frac{a_{0}}{2}+\sum_{k = 1}^{\infty} (a_{k} \cdot cos(k \omega t)+ b_{k} \cdot sin(k \omega t)) = f_{AV} + \sum_{k = 1}^{\infty} c_k \cdot \sin(k\omega t + \varphi_k)$$

Die Koeffizienten der Entwicklung von $f(t)$ sind:

\begin{tabular}{ll}
  $a_{0} = \frac{2}{T}\int\limits_{0}^{T}f(t)dt$ & \\
  $a_{k} = \frac{2}{T}\int\limits_{0}^{T}f(t) \cdot cos(k \omega t)dt$   &\ $(k = 0,1,2,...)$\\
  $b_{k} = \frac{2}{T}\int\limits_{0}^{T}f(t) \cdot sin(k \omega t)dt$   &\ $(k = 1,2,3,...)$\\
  $c_{k} = \sqrt{a_k^2 + b_k^2},\quad \varphi_k = \arctan(\frac{b_k}{a_k})$
\end{tabular}

\subsection{Komplexe Darstellung der Fourierreihen}

$$f(t) = \sum\limits_{k = -\infty}^{\infty} c_k \cdot e^{j k \omega t} \qquad \text{mit} \qquad c_n=\overline{c_{-n}}=\frac{1}{T}\int\limits_0^T{f(t)\cdot e^{-jn\omega t}dt}$$



\subsubsection{Umrechnungsformeln}
	$$c_n=\overline{c_{-n}}=\frac{a_n-jb_n}{2} (n=0,1,2,3,\ldots\text{ wobei }b_0=0)\qquad
      \left.
      \begin{array}{l} 
        a_n=2 \cdot \text{Re}(c_n)\\
        b_n=-2 \cdot \text{Im}(c_n)
      \end{array}
        \right\} 
        \quad
      (n=0,1,2,3,\ldots, b_0 = 0)$$
\subsection{Sätze zur Berechnung der Fourierkoeffizienten}
\subsubsection{Symmetrie}
\begin{tabular}{ll}
  Falls $f(t)$ gerade ist $(f(t) = f(-t))$: &\ $\rightarrow b_{n} = 0, a_{n} = \frac{4}{T}\int\limits_{0}^{\frac{T}{2}} f(t) \cdot cos(n \omega t)dt$\\
  Falls $f(t)$ ungerade ist $(f(-t) = -f(t))$: &\ $\rightarrow a_{n} = 0, b_{n} = \frac{4}{T}\int\limits_{0}^{\frac{T}{2}} f(t) \cdot sin(n \omega t)dt$\\
\end{tabular}\\\\
Beispiele von geraden Funktionen sind $x^2, cos(x)$ und Beispiele ungerader Funktionen sind $x, x^3, sin(x)$.
