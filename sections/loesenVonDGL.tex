
\section{Lösen von Differentialgleichungen}
    \subsection{Allgemeine Vorgehensweisen}
        \subsubsection{Trennung von Variablen / Separation}
            \begin{tabular}{p{4cm}p{1.5cm}p{10.5cm}}
                \textbf{Form:} $y' = f(x) g(y)$ &
                \textbf{Vorgehen:}              &
                1. DGL umstellen: $\frac{y'}{g(y)} = f(x)$ \\ &&
                2. Beidseitig nach x integrieren wobei $dx = \frac{dy}{y'}$ \\ &&
                3. Grenzen anpassen: $\int\limits_{y_0=y(x_0)}^{y} \frac{1}{g(y)} dy =
                \int\limits_{x}^{x_0}f(x) dx$
            \end{tabular}
            
        \subsubsection{Lineartermsubstitution}
            \begin{tabular}{p{4cm}p{1.5cm}p{10.5cm}}
                \textbf{Form:} $y'=f(ax+by+c)$   &
                \textbf{Vorgehen:}               &
                1. Substitution: $z=ax+by+c$ \\ &&
                3. Einsetzen in $z'=a+bf(z)$\\ &&
                2. Separation: $\frac{z'}{f(z)} = a + b$ wobei $z_0 = x_0 + y_0$
            \end{tabular}
                
        \subsubsection{Gleichgradigkeit}
            \begin{tabular}{p{4cm}p{1.5cm}p{10.5cm}}
                \textbf{Form:} $y'=f(\frac{y}{x})$ &
                \textbf{Vorgehen:}                &
                1. Substitution:\quad $z=\frac{y}{x}$\\ &&
                2. Einsetzen in $z'=\frac{1}{x}(f(z)-z)$\\ &&
                3. Separation: $\frac{z'}{f(z)-z} = \frac{1}{x}$ wobei $z_0 = \frac{y_0}{x_0}$ 
            \end{tabular}
            
    \subsection{Differentialgleichung 1. Ordnung}
        \subsubsection{konstante Störung f(x) = C}
            \textbf{1.} Homogene Lösung mit $\mathrm{y_h = 0}$ berechnen\\
            \textbf{2.} Partikuläre Lösung mit $\mathrm{y_p = C}$ (= Konstante) berechnen, indem zeitlich abhängige Terme der DGL ignoriert werden
        \subsubsection{sinusförmige Störung f(x) = $\mathrm{\bf U \cdot \sin(\omega t)}$}
            \textbf{1.} Homogene Lösung mit $\mathrm{y_h = 0}$ berechnen\\
            \textbf{2.} Partikuläre Lösung mit $\mathrm{y_p = C \cdot \sin(\omega t) + D\cdot \cos(\omega t)}$ berechnen, indem $\mathrm{y_p}$ in DGL eingesetzt wird und mittels Koeffizientenvergleich die Vorfaktoren der Sinus- und der Cosinusschwingung bestimmt werden. Für die Vorfaktoren der Sinus- und Cosinusschwingung je eine Gleichung aufstellen und das Gleichungssystem der beiden Gleichungen lösen.