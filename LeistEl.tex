
\newcommand{\titleinfo}{Leistungselektronik  - Formelsammlung}
\newcommand{\authorinfo}{S\"ami Bertsch}
\newcommand{\versioninfo}{ }

% Genereller Header
\documentclass[10pt,twoside,a4paper,fleqn]{article}
\usepackage[utf8]{inputenc}
\usepackage[left=1cm,right=1cm,top=1cm,bottom=1cm,includeheadfoot]{geometry}
\usepackage[ngerman]{babel,varioref}
\usepackage{booktabs}
\usepackage{longtable}
\usepackage{tabularx}
%\usepackage{wasysym}
%\usepackage[demo]{graphicx}

% Pakete
\usepackage{amssymb,amsmath,fancybox,graphicx,color,lastpage,wrapfig,fancyhdr,hyperref,verbatim,floatflt}


%%%%%%%%%%%%%%%%%%%%
% Generelle Makros %
%%%%%%%%%%%%%%%%%%%%
\newcommand{\skript}[1]{$_{\textcolor{red}{\mbox{\small{S#1}}}}$}
\newcommand{\verweis}[2]{\small{(siehe auch \ref{#1}, #2 (S. \pageref{#1}))}}
\newcommand{\subsubadd}[1]{\textcolor{black}{\mbox{#1}}}


\newcolumntype{L}[1]{>{\raggedright\arraybackslash}p{#1}} % linksbündig mit Breitenangabe
\newcolumntype{C}[1]{>{\centering\arraybackslash}p{#1}} % zentriert mit Breitenangabe
\newcolumntype{R}[1]{>{\raggedleft\arraybackslash}p{#1}} % rechtsbündig mit Breitenangabe

\newcommand{\ltab}{\raggedright\arraybackslash} % Tabellenabschnitt linksbündig
\newcommand{\ctab}{\centering\arraybackslash} % Tabellenabschnitt zentriert
\newcommand{\rtab}{\raggedleft\arraybackslash} % Tabellenabschnitt rechtsbündig



\newcommand{\skriptsection}[2]{\section{#1 {\tiny Vorlesung #2}}}
\newcommand{\skriptsubsection}[2]{\subsection{#1 {\tiny Folie #2}}}
\newcommand{\skriptsubsubsection}[2]{\subsubsection{#1 {\tiny Folie #2}}}


%%%%%%%%%%
% Farben %
%%%%%%%%%%
\definecolor{black}{rgb}{0,0,0}
\definecolor{red}{rgb}{1,0,0}
\definecolor{white}{rgb}{1,1,1}
\definecolor{grey}{rgb}{0.8,0.8,0.8}
\definecolor{hellgrau}{rgb}{0.95,0.95,0.95}

%%%%%%%%%%%%%%%%%%%%%%%%%%%%
% Mathematische Operatoren %
%%%%%%%%%%%%%%%%%%%%%%%%%%%%
\DeclareMathOperator{\sinc}{sinc}

\newcommand{\zaehlerBuck}{\frac{1}{i_{Lmax}} \cdot \frac{V_{1}}{R_{1}} - 1   }
\newcommand{\nennerBuck}{\frac{1}{i_{Lmax}} \cdot \frac{V_{1}}{R_{1}}-e^{\frac{-T_{a}}{\tau}} }

% Fouriertransformationen
\unitlength1cm
\newcommand{\FT}
{
\begin{picture}(1,0.5)
\put(0.2,0.1){\circle{0.14}}\put(0.27,0.1){\line(1,0){0.5}}\put(0.77,0.1){\circle*{0.14}}
\end{picture}
}


\newcommand{\IFT}
{
\begin{picture}(1,0.5)
\put(0.2,0.1){\circle*{0.14}}\put(0.27,0.1){\line(1,0){0.45}}\put(0.77,0.1){\circle{0.14}}
\end{picture}
}



%%%%%%%%%%%%%%%%%%%%%%%%%%%%
% Allgemeine Einstellungen %
%%%%%%%%%%%%%%%%%%%%%%%%%%%%
%pdf info
\hypersetup{pdfauthor={\authorinfo},pdftitle={\titleinfo},colorlinks=false}
\author{\authorinfo}
\title{\titleinfo}

%Kopf- und Fusszeile
\pagestyle{fancy}
\fancyhf{}
%Linien oben und unten
\renewcommand{\headrulewidth}{0.5pt} 
\renewcommand{\footrulewidth}{0.5pt}

\fancyhead[L]{\titleinfo{ }}
%Kopfzeile rechts bzw. aussen
\fancyhead[R]{Seite \thepage { }von \pageref{LastPage}}
%Fusszeile links bzw. innen
%\fancyfoot[L]{\footnotesize{\authorinfo}}
%Fusszeile rechts bzw. ausen
\fancyfoot[R]{\footnotesize{\today}}

% Einrücken verhindern versuchen
\setlength{\parindent}{0pt}

\begin{document}

\section{Allgemeine Formeln}

\begin{tabular}{ll}
  Spannung über einer Induktivität &\ $u_{L}(t) = L\frac{di}{dt}$\\
  Strom durch Kondensator &\ $i_{C}(t) = C\frac{du}{dt}$\\
  Zeitkonstante $\tau$ &\ $\tau = \frac{L}{R}$ oder $\tau = RC$\\
  Berechnung des Mittelwertes &\ $X_{AV} = \frac{1}{T}\int\limits_{0}^{T}x(t)dt$\\
  Berechnung des Gleichwertes &\ $\overline{|X|} = \frac{1}{T} \int\limits_{0}^{T} |x(t)|dt$\\
  Berechnung des Effektivwertes &\ $X_{RMS} = \sqrt{\frac{1}{T}\int\limits_{0}^{T}x^2(t)dt}$\\
\end{tabular}

\input{./sections/berechnungHoehererHarmonischen}

\section{Ungesteuerter Gleichrichter M1U}


\begin{tabu}{|l|l|m{0.3\textwidth}}
\cline{1-2}
  Mittelwert
  	& $ \begin{aligned}
  			U_{R AV} &= \frac{1}{T}\int\limits_{0}^{T}u_{R}(t)dt = \frac{1}{2\pi}\int\limits_{0}^{\pi}\hat{U}_{2} \cdot sin(\alpha)d\alpha, \qquad \alpha = \omega t\\
  					&= -\frac{\hat{U}_{2}}{2\pi}(cos(\pi)-cos(0))= \frac{\hat{U}_{2}}{\pi}
  		\end{aligned}$
  	& \multirow{7}{*}[4cm]{\includegraphics[width = \linewidth]{./pictures/m1u.png}}\\
  	\cline{1-2}
  bei f = 50 Hz 
  	& $U_{R AV} = \frac{1}{20 ms}\int\limits_{0}^{10 ms}u_{R}(t)dt, T = \frac{1}{f}$ &\\
  	\cline{1-2}
  Effektivwert 
  	& $U_{R RMS} = \sqrt{\frac{1}{T}\int\limits_{0}^{T}u_{R}^2 \cdot dt} = \sqrt{\frac{1}{2\pi}\int\limits_{0}^{\pi}\hat{U}_{2}^2 \cdot sin^2(\alpha) \cdot d\alpha} = \frac{\hat{U}_{2}}{2}$ &\\
  	\cline{1-2}
  Allgemein 
  	& $\int\limits_{0}^{\pi}sin^2(\alpha) \cdot d\alpha = \frac{\pi}{2}$&\\
  	\cline{1-2}
  bei f = 50 Hz 
  	& $U_{R RMS} = \sqrt{\frac{1}{20 ms}\int\limits_{0}^{10 ms}u_{R}^2 \cdot dt}$&\\
  	\cline{1-2}
  Laststrom 
  	& $i_{R}(t) = \frac{u_{R}(t)}{R}$&\\
  	\cline{1-2}
  Wirkleistung 
  	& $P = \frac{1}{2\pi}\int\limits_{0}^{\pi}\frac{u_{R}^2(\alpha)}{R} \cdot d\alpha = \frac{U_{R RMS}^2}{R} = \frac{\hat{U}_{2}^2}{4R}$&\\
\cline{1-2}
\end{tabu}


\section{Gesteuerter Gleichrichter M1C}
der Steuerwinkel des Gleichrichter: $\alpha \in [0, \pi]$\\\\
\begin{tabular}{ll}
  Mittelwert &\ $U_{R AV} = \frac{1}{2\pi}\int\limits_{\alpha}^{\pi}U_{2m} \cdot sin(\beta) \cdot d\beta, \beta = \omega t$\\
  &\ $U_{R AV} = \frac{U_{2m}}{2\pi}(1 + cos(\alpha))$\\\\
  Effektivwert &\ $U_{R RMS} = \sqrt{\frac{U_{2m}^2}{2\pi}\int\limits_{\alpha}^{\pi}sin^2(\beta) \cdot d\beta} = U_{2m} \cdot \sqrt{\frac{\pi-\alpha}{4\pi}+\frac{sin(2\alpha)}{8\pi}}$\\
  allgemein &\ $\int\limits_{\alpha}^{\pi}sin^2(\beta) \cdot d\beta = \frac{\pi-\alpha}{2}+\frac{sin(2\alpha)}{4}$\\\\
\end{tabular}

\section{Gesteuerter Gleichrichter B2C}
\begin{tabular}{ll}
  Mittelwert &\ $U_{R AV} = \frac{1}{\pi}\int_{\alpha}^{\pi}U_{2m} \cdot sin(\beta) \cdot d\beta, \beta = \omega t$\\
  &\ $U_{R AV} = \frac{U_{2m}}{\pi}(1 + cos(\alpha))$\\\\
  Effektivwert &\ $\sqrt{\frac{U_{2m}^2}{\pi}\int_{\alpha}^{\pi}sin(\beta)^2 \cdot d\beta} = U_{2m} \cdot \sqrt{\frac{\pi-\alpha}{2\pi}+\frac{sin(2\alpha)}{4\pi}}$\\\\
\end{tabular}

\section{Leistungsberechnung}
\begin{tabu}{|p{0.3\textwidth}|p{0.6\textwidth}|}
  \hline
  Momentane Leistung
  	& $p(t) = u_{R}(t) \cdot i_{R}(t)$\\
  \hline
  Wirkleistung
  	& $P = \frac{1}{2\pi}\int\limits_{0}^{\pi}\frac{u_{R}^2(\alpha)}{R} \cdot d\alpha = \frac{U_{R RMS}^2}{R}$\\
  \hline
  Wirkleistung (Trafoseitig)
  	& $P = U_{RMS} \cdot I_{1} \cdot cos(\varphi_{1})$ \newline
  		dabei ist $I_{1}$ die erste Harmonische Komponente des Stromes \newline
  		und $\varphi_{1}$ die Phasenverschiebung\\
  \hline
  Grundschwingungsblindleistung
  	& $Q_{1} = U_{RMS} \cdot I_{1} \cdot sin(\varphi_{1})$\\
  \hline
  Verzerrungsleistung
  	& $Q_{V} =  U_{RMS} \cdot \sqrt{\sum\limits_{k = 2}^{\infty}I_{k}^2}$\\
  \hline
  gesamte Blindleistung
  	& $Q = \sqrt{Q_{1}^2 + Q_{V}^2}$\\
  \hline
  Grundschwingungsscheinleistung
  	& $S_{1} = U_{RMS} \cdot I_{1}$\\
  \hline
  gesamte Scheinleistung
  	& $S = U_{RMS} \cdot I_{RMS} = \sqrt{P^2 + Q^2} = \sqrt{P^2 + Q_{1}^2 + Q_{V}^2}$\\
  \hline
  Leistungsfaktor
  	& $\lambda = \frac{P}{S}$\\
  \hline
\end{tabu}

\section{Schaltverluste, Kühlung}
B2C als Beispiel:\\\\
in einem ersten Schritt muss der Strom durch den Thyristor berechnet werden:\\
\begin{tabular}{ll}
  &\ $I_{Rm} = \frac{U_{2m}}{R}$\\\\
  Mittelwert des Thyristorstroms &\ $I_{T AV} = \frac{1}{2\pi}\int_{\alpha}^{\pi}I_{Rm} \cdot sin(\beta) \cdot d\beta, \beta = \omega t$\\
  &\ $I_{T AV} = \frac{I_{Rm}}{2\pi} \cdot (1+cos\alpha)$\\\\
  Effektivwert des Thyristorstroms &\ $I_{T RMS} = \sqrt{\frac{I_{RM}^2}{2\pi}\int_{\alpha}^{\pi}sin^2(\beta)d\beta}$\\
  &\ $I_{T RMS} = \frac{I_{Rm}}{2}\sqrt{\frac{\pi - \alpha}{\pi}+\frac{sin2\alpha}{2\pi}}$\\\\
\end{tabular}

\begin{figure}[htbp]
  \begin{minipage}[t]{6cm}
    \vspace{0pt}
    \centering
    \includegraphics[width = 5cm]{./pictures/kennlinieThyristor} 
  \end{minipage}
  \hfill
  \begin{minipage}[t]{6cm}
    \vspace{0pt}
    Durchlassrichtung: $i_{T}, u_{T}$\\
    Schwellenspannung: $U_{T0}$\\
    Differentieller Durchlasswiderstand: $r_{T} = \dfrac{du_{T}}{di_{T}}$\\
    \includegraphics[width = 4cm]{./pictures/schemaThyristor}\\
    $u_{T} = U_{T0}+i_{T}(t) \cdot r_{T}$
  \end{minipage}
\end{figure}

momentane Verlustleistung: $p(t) = u_{T}(t) \cdot i_{T}(t)$\\
\begin{tabular}{ll}
  Mittelwert der Verlustleistung: &\ $P_{T} = \frac{1}{T}\int_{0}^{T}u_{T}(t) \cdot i_{T}(t) \cdot dt$\\
  &\ $P_{T} = U_{T0} \cdot \frac{1}{T}\int_{0}^{T}i_{T}(t)dt+r_{T} \cdot \frac{1}{T}\int_{0}^{T}i_{T}^2(t)dt$\\\\
  &\ $P_{T} = U_{T0} \cdot I_{T AV} + r_{T} \cdot I_{T RMS}^2$\\
  &\ $I_{T AV}$ ist der Mittelwert und $I_{T RMS}$ der\\ &\ Effektivwert des Thyristorstroms\\
\end{tabular}

\textbf{Die Werte für $U_{T0}$ können aus dem Datenblatt des Thyristors herausgelesen werden.}\\\\\\
\begin{tabular}{|l|l|}
  \hline
  \textbf{Thermische Kenngrössen} &\ \textbf{Elektrische Kenngrössen}\\
  \hline
  Wärmeleistung $ P (W) $ &\ Strom $ I (A) $\\
  \hline
  Temperaturunterschied $\vartheta (K)$ &\ Spannung $U (V)$\\
  \hline
  Wärmewiderstand $R_{th} (\frac{K}{W})$ &\ Widerstand $R (\frac{V}{A})$\\
  \hline
\end{tabular}

\subsection{Thyristor ohne Kühlkörper}
\begin{figure}[htbp]
  \begin{minipage}[t]{6cm}
    \vspace{0pt}
    \centering
    \includegraphics[width = 5cm]{./pictures/ohneKuehlkoerper} 
  \end{minipage}
  \hfill
  \begin{minipage}[t]{6cm}
    \vspace{0pt}
    $\vartheta_{vJ} - \vartheta_{U} = P \cdot (R_{th JG}+ R_{th GU})$\\
    $ \vartheta_{vj} = P \cdot (R_{th JG}+ R_{th GU}) + \vartheta_{U}$\\\\\\
    \textbf{$R_{th}$ muss wiederum aus dem Datenblatt herausgelesen werden.}
  \end{minipage}
\end{figure}

\subsection{Thyristor mit Kühlkörper}
\begin{figure}[htbp]
  \begin{minipage}[t]{6cm}
    \vspace{0pt}
    \centering
    \includegraphics[width = 5cm]{./pictures/mitKuehlkoerper} 
  \end{minipage}
  \hfill
  \begin{minipage}[t]{6cm}
    \vspace{0pt}
    $\vartheta_{vJ} - \vartheta_{U} = P \cdot (R_{th JG}+ R_{th GK}+ R_{th KU})$\\
    $ \vartheta_{vj} = P \cdot (R_{th JG}+ R_{th Gk} +R_{th KU}) + \vartheta_{U}$\\\\\\
    
  \end{minipage}
\end{figure}


\section{Gleichstromumrichter}
\subsection{Buck-Converter (Tiefsetzsteller)}
Ein einfacher Tiefsetzsteller könnte auch mit einem Spannungsteiler bebaut werden.
Die Verlustleistung würde jedoch $P_{V} = R_{1} \cdot I_{1}^2$ betragen.


Grundgleichungen:\\
$V_{1}= i_{L} \cdot R + L\dfrac{di_{L}}{dt}, t \in [0; T_{e}]$\\
$0 = i_{L} \cdot R + L\dfrac{di_{L}}{dt}, t \in [T_{e}; T_{s}]$ ($L\dfrac{di_{L}}{dt}$ ist dabei die Spannung, welche die Induktivität abgibt
$\rightarrow$ Quelle in diesem Fall)\\\\

\begin{minipage}{ 0.5 \linewidth}
  Durch das Lösen der Grundgleichungen erhält man die den Verlauf des Stromes:\\\\
  
  $i_{L} = \frac{V_{1}}{R_{1}}+\frac{V_{1}}{R_{1}} \cdot \frac{-1+e^{\frac{-T_{a}}{\tau}}}{1-e^{\frac{-T_{s}}{\tau}}} \cdot e^{\frac{-t}{\tau}}, t \in [0; T_{e}]$\\\\
  $i_{L} = \frac{V_{1}}{R_{1}} \cdot \frac{1-e^{\frac{-T_{e}}{\tau}}}{1-e^{\frac{-T_{s}}{\tau}}} \cdot e^{\frac{-(t+T_{e})}{\tau}}, t \in [T_{e}; T_{s}]$ mit $\tau = \frac{L_{1}}{R_{1}}$\\\\
  und folgende Ein- und Ausschaltzeiten:\\\\
  $T_{a} = -\tau \cdot \ln\frac{i_{Lmin}}{i_{Lmax}}$ mit $\tau = \frac{L_{1}}{R_{1}}$\\\\
  $T_{e} = -\tau \cdot \ln\left(\frac{\zaehlerBuck}{\nennerBuck}\right)$\\\\
  $T_{s} = T_{e} + T_{a}$
\end{minipage}
\hfill
\begin{minipage}{0.5 \linewidth}
  \includegraphics[width = \textwidth]{./pictures/buckConverter}
\end{minipage}


\input{./sections/gleichstromschalter}

\end{document}